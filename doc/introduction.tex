%%%%%%%%%%%%%%%%%%%%%%%%%%%%%%%%%%%%%%%%%%%%%%%%%%%%%%
% Filename:      introduction.tex
% Author:        Junwei Wang(wakemecn@gmail.com)
% Last Modified: 2012-04-16 14:12
% Description:
%%%%%%%%%%%%%%%%%%%%%%%%%%%%%%%%%%%%%%%%%%%%%%%%%%%%%%
\chapter{绪论}
\section{基于属性加密的研究背景}
在许多情况下,当用户对敏感数据进行加密时,这时他迫切需要建立特定
的访问控制策略决定谁可以解密数据。例如,假设FBI在Knoxvile和San
Francisco的公共反腐办公室调查涉及San Francisco说客行贿Tennesseei
国会议员的指控。上级FBI人员可能想加密一份敏感的备忘录以便只有拥有
特定凭据和属性的人员可以查看。例如,上级人员可能指定下述访问结构
来访问此信息:
\centerline{\textsc{(("PUBLIC CORRUPTION OFFICE" \textbf{AND} 
("KNOXVILLE" }}
\centerline{\textsc{\textbf{OR} "SAN FRANCISCO"))
\textbf{OR} "NAME: CHARLIE EPPES")}} 
通过这个访问结构,上级人员可能意味着备忘录只能
被工作在公共Knowxille或者San Francisco反腐办公室的或者名字叫做
Charlie Eppes的人员查看。\par
就像上例中所解释的,秘密数据拥有者能够根据特定的只是选择访问策略。
除此之外,这个人可能不知道其他所有能够访问被加密数据人员的精确身
份,但至少他有以属性或者凭据的方式描述这些人的方法。\par
传统方案中,这种表达式形式的访问控制必须采用可信服务器本地存储数
据的方式。服务器作为通过检查用户是否拥有某种凭据来控制访问记录或
文件的检查器被信任。然而,随着数据的增长,数据会被存储到越来越多
服务器的分布式环境中。在分布式环境中复制数据具有高性能和高可靠性
的优势。但是随着数据的增长,保证数据安全性越来越难;当数据存储在
多个服务器上,任一这些服务器被攻破的可能性大幅攀升。基于这些原因
我们需要敏感数据以加密的方式存储,以便在服务器被攻破的情况下数据
仍然保密。\par
公钥加密是一个保障存储和传输敏感数据的强有力机制。传统上,加密被
看作是用户向目标用户或者设备分享数据的一种方法。然而这种方法仅适
用于数据提供方精确的知道确切的知道要跟谁分享数据,然而更多的应用
中数据提供者想要更具接受者的凭据来制定分享机制。也就是说,大多数
现存的公钥加密方案允许一方给另一方加密数据,但是不能够高效的处理
像上述例子中更复杂的表达式形式的访问控制。\par
\section{国内外研究现状}
Sahai和Waters\cite{my}
\section{论文的组织结构}
