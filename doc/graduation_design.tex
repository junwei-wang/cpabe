%%%%%%%%%%%%%%%%%%%%%%%%%%%%%%%%%%%%%%%%%%%%%%%%%%%%%%
% Filename:      graduation_design.tex
% Author:        Junwei Wang(wakemecn@gmail.com)
% Last Modified: 2012-05-06 11:07
% Description:
%%%%%%%%%%%%%%%%%%%%%%%%%%%%%%%%%%%%%%%%%%%%%%%%%%%%%%
\documentclass[a4paper,12pt,oneside,openany]{book}
\usepackage{ctex}
\usepackage[top=1in,bottom=1in,left=1.25in,right=1.25in]{geometry}
\usepackage{graphicx}
\usepackage{ulem}
\usepackage{amsfonts,amssymb}
\usepackage{amsmath}
\usepackage{hyperref}
\usepackage[bf,small,center,indentafter,pagestyles]{titlesec}
% redefine some name
\renewcommand{\chaptername}{第\chinese{chapter}章}
\renewcommand{\bibname}{参考文献}
% redefine the chapter section
\titleclass{\chapter}{straight}
\titleformat{\chapter}[hang]
{\centering\zihao{-3}\bfseries}{第\chinese{chapter}章}{1em}{}
\titleformat{\section}
{\normalfont\zihao{4}\bfseries}{\thesection}{11pt}{\large}
% define pagehead style
\newpagestyle{firstpage}{
}
\newpagestyle{abstract}{
    \sethead{}{}{\kaishu\small\hfill 摘\qquad 要\hfill \thepage}
    \setfoot{}{}{}\headrule
}
\newpagestyle{tableofcontent}{
    \sethead{}{}{\kaishu\small\hfill 目\qquad 录 \hfill \thepage}
    \setfoot{}{}{}\headrule
}
\newpagestyle{main}{
   \sethead{}{}{\kaishu\small\hfill \chaptername\quad\chaptertitle\hfill\thepage}
   \setfoot{}{}{}\headrule
}
%%
\usepackage{titletoc}
\usepackage[titletoc]{appendix}
%定义目录格式
\titlecontents{chapter}[0em]{\normalsize}
{\thecontentslabel\hspace*{1em}}{\hspace*{-2.3em}}
{\titlerule*[0.8pc]{.}\contentspage}

\titlecontents{section}[0em]{\normalsize}
{\thecontentslabel\hspace*{1em}}{\hspace*{-2.3em}}
{\titlerule*[0.8pc]{.}\contentspage}

\titlecontents{subsection}[0em] {\normalsize}
{\thecontentslabel\hspace*{1em}}{\hspace*{-2.3em}}
{\titlerule*[0.8pc]{.}\contentspage}
%%%%%%%%%%%%%code style%%%%%%%%%%
\usepackage{xcolor}
\renewcommand{\ttdefault}{phv}
\usepackage{listings}
\renewcommand{\lstlistingname}{代码}
\lstset{
         basicstyle=\footnotesize\ttfamily, % Standardschrift
         %numbers=left,               % Ort der Zeilennummern
         numberstyle=\tiny,          % Stil der Zeilennummern
         %stepnumber=2,               % Abstand zwischen den Zeilennummern
         numbersep=5pt,              % Abstand der Nummern zum Text
         tabsize=2,                  % Groesse von Tabs
         extendedchars=true,         %
         breaklines=true,            % Zeilen werden Umgebrochen
         keywordstyle=\color{red},
            frame=b,         
         stringstyle=\color{white}\ttfamily, % Farbe der String
         showspaces=false,           % Leerzeichen anzeigen ?
         showtabs=false,             % Tabs anzeigen ?
         xleftmargin=17pt,
         framexleftmargin=17pt,
         framexrightmargin=5pt,
         framexbottommargin=4pt,
%        backgroundcolor=\color{lightgray},
         showstringspaces=false      % Leerzeichen in Strings anzeigen ?       
}
\lstloadlanguages{% Check Dokumentation for further languages ...
         %[Visual]Basic
         %Pascal
         %C
         %C++
         %XML
         %HTML
         Java
}
\usepackage{caption}
\DeclareCaptionFont{white}{\color{white}}
\DeclareCaptionFormat{listing}{\colorbox[cmyk]{0.43, 0.35, 0.35,0.01}{\parbox{\textwidth}{#1#2#3}}}
\captionsetup[lstlisting]{format=listing,justification=raggedright,labelfont=white,textfont=white, singlelinecheck=false, margin=0pt, font={sf,bf,footnotesize}}
%%%%%%%%%%%%%code style%%%%%%%%%%
% 取消中文连字符与文字间的空格
\normalspacedchars{-}
% 定义"defination"环境
\newtheorem{Definition}{定义}
\begin{document}
% titlepage
\frontmatter
\pagestyle{firstpage}
%%%%%%%%%%%%%%%%%%%%%%%%%%%%%%%%%%%%%%%%%%%%%%%%%%%%%%
% Filename:      titlepage.tex
% Author:        Junwei Wang(wakemecn@gmail.com)
% Last Modified: 2012-04-17 09:03
% Description:
%%%%%%%%%%%%%%%%%%%%%%%%%%%%%%%%%%%%%%%%%%%%%%%%%%%%%%
\vspace{30mm}
\begin{center}
% the picture
\includegraphics[scale=0.7]{sdu_pic.jpg}\\
\zihao{0} 毕业论文(设计)\\
%
\vspace{15mm}
\leftline{\heiti\zihao{-2} 设计(论文)题目:}
\heiti\zihao{-2} 密文策略--基于属性的加密\par
%
\vspace{60mm}
\zihao{4}姓\qquad 名\uline{\kaishu\makebox[50mm]{王军委}}\\
\zihao{4}学\qquad 号\uline{\kaishu\makebox[50mm]{200800300237}}\\
\zihao{4}学\qquad 院\uline{\kaishu\makebox[50mm]{软件学院}}\\
\zihao{4}专\qquad 业\uline{\kaishu\makebox[50mm]{软件工程}}\\
\zihao{4}年\qquad 级\uline{\kaishu\makebox[50mm]{2008级}}\\
\zihao{4}指导老师\uline{\kaishu\makebox[50mm]{胡程瑜}}\\
%
\vspace{5mm}
\zihao{4}\today
\end{center}

% abstract
\pagestyle{abstract}
%%%%%%%%%%%%%%%%%%%%%%%%%%%%%%%%%%%%%%%%%%%%%%%%%%%%%%
% Filename:      abstract.tex
% Author:        Junwei Wang(wakemecn@gmail.com)
% Last Modified: 2012-05-06 10:45
% Description:
%%%%%%%%%%%%%%%%%%%%%%%%%%%%%%%%%%%%%%%%%%%%%%%%%%%%%%
\centerline{\zihao{-2}\bfseries 密文策略--基于属性的加密}
\addcontentsline{toc}{chapter}{\qquad 摘\qquad 要}
\chapter*{摘\qquad 要}
在分布式系统中,只有当用户拥有某些凭据或者属性才能够访问数据。
目前,执行这种策略的唯一方式是采用一个可信的服务器来存储数据
并调整访问控制。然而,如果任一存储数据的服务器被攻破,那么数
据的保密性就会大打折扣。在本文中,我们提供一个称之为密文策略
--基于属性的加密来实现对加密数据的复杂的访问控制。通过我们的
技术,加密数据甚至可以存储在不可信的存储服务器上。此外,我们
的方法可以抵制合谋攻击。先前的基于属性的加密方案用属性来描述
被加密的数据,而将策略写入用户的的密钥里;然而在我们的系统中
属性用来描述用户的凭据,加密数据的一方来决定谁可以解密的策略
。因此,我们的加密方案在概念上更接近与传统的访问控制方法,如
基于角色的访问控制(Role-Base Access Control,RBAC)。\par
除此之外,我们提供了我们系统的实现和性能测试。\\
\textbf{关键字:} 访问控制;合谋攻击;密文策略;基于属性的加密
\addcontentsline{toc}{chapter}{\qquad ABSTRACT}
\chapter*{ABSTRACT}
In several distributed systems a user should only be %
able to access data if a user possesses a certain set of %
credentials or attributes. Currently, the only method %
for enforcing such policies is to employ a trusted server %
to store the data and mediate access control. However, %
if any server storing the data is compromised, then the %
confidentiality of the data will be compromised. In this %
paper we present a system for realizing complex access %
control on encrypted data that we call Ciphertext-Policy %
Attribute-Based Encryption. By using our techniques %
encrypted data can be kept confidential even if the %
storage server is untrusted; moreover, our methods are %
secure against collusion attacks. Previous Attribute-Based %
Encryption systems used attributes to describe %
the encrypted data and built policies into user’s keys;%
while in our system attributes are used to describe a %
user’s credentials, and a party encrypting data determines %
a policy for who can decrypt. Thus, our methods %
are conceptually closer to traditional access control %
methods such as Role-Based Access Control (RBAC).\par
In addition, we provide an implementation of our system %
and give performance measurements.\\
\textbf{Keyword:}Access Control;Collusion Attacks;%
Ciphertext-Policy;Attribute-Based Encryption

% contents
\pagestyle{tableofcontent}
\renewcommand{\contentsname}{目\quad 录}
\tableofcontents
% 绪论 
\mainmatter
\pagestyle{main}
%%%%%%%%%%%%%%%%%%%%%%%%%%%%%%%%%%%%%%%%%%%%%%%%%%%%%%
% Filename:      introduction.tex
% Author:        Junwei Wang(wakemecn@gmail.com)
% Last Modified: 2012-04-17 10:18
% Description:
%%%%%%%%%%%%%%%%%%%%%%%%%%%%%%%%%%%%%%%%%%%%%%%%%%%%%%
\chapter{绪论}
\section{基于属性的加密研究背景}
在许多情况下,当用户对敏感数据进行加密时,这时他迫切需要建立特定
的访问控制策略决定谁可以解密数据。例如,假设FBI在Knoxvile和San
Francisco的公共反腐办公室调查涉及San Francisco说客行贿Tennesseei
国会议员的指控。上级FBI人员可能想加密一份敏感的备忘录以便只有拥有
特定凭据和属性的人员可以查看。例如,上级人员可能指定下述访问结构
来访问此信息:
\centerline{\textsc{(("Public Corruption Office" \textbf{AND} 
("Knoxville" }}
\centerline{\textsc{\textbf{OR} "San Francisco"))
\textbf{OR} "Name: Charlie Eppes")}} 
通过这个访问结构,上级人员可能意味着备忘录只能
被工作在公共Knowxille或者San Francisco反腐办公室的或者名字叫做
Charlie Eppes的人员查看。\par
就像上例中所解释的,秘密数据拥有者能够根据特定的只是选择访问策略。
除此之外,这个人可能不知道其他所有能够访问被加密数据人员的精确身
份,但至少他有以属性或者凭据的方式描述这些人的方法。\par
传统方案中,这种表达式形式的访问控制必须采用可信服务器本地存储数
据的方式。服务器作为通过检查用户是否拥有某种凭据来控制访问记录或
文件的检查器被信任。然而,随着数据的增长,数据会被存储到越来越多
服务器的分布式环境中。在分布式环境中复制数据具有高性能和高可靠性
的优势。但是随着数据的增长,保证数据安全性越来越难;当数据存储在
多个服务器上,任一这些服务器被攻破的可能性大幅攀升。基于这些原因
我们需要敏感数据以加密的方式存储,以便在服务器被攻破的情况下数据
仍然保密。\par
公钥加密是一个保障存储和传输敏感数据的强有力机制。传统上,加密被
看作是用户向目标用户或者设备分享数据的一种方法。然而这种方法仅适
用于数据提供方精确的知道确切的知道要跟谁分享数据,然而更多的应用
中数据提供者想要更具接受者的凭据来制定分享机制。也就是说,大多数
现存的公钥加密方案允许一方给另一方加密数据,但是不能够高效的处理
像上述例子中更复杂的表达式形式的访问控制。\par
\section{国内外研究现状}
Sahai和Waters\cite{SW:FuzzyIBE}提出了加密算法本身决定谁可以解密被加
密的加密新思路。数据提供者用谓词$f(\cdot)$描述他想他想如何分享数
据,用户被赋予与他们的凭据$X$关联的密钥;当$f(X)=1$时,拥有凭据
$X$的用户可以解密拥有谓词$f$的密文。Sahai和Waters\cite{SW:FuzzyIBE}
提出了这种问题的正式描述,他们称之为基于属性的加密\footnote{Attribute-Based 
Encryption, 简称ABE}。在ABE中,用户的平局被称为“属性”的字符
串集合表示,谓词用对这些属性的公式表示。SW\footnote{Amit Sahai和
Brent Waters}的技术受到基于身份的加密\footnote{Identity-Based 
Encryption,以下IBE}上一些工作\cite{Shamir:IBC,BF:IBE,Cocks:IBE,CHK:fspk,BB:IBE}
的启发。SW方法的一个缺点是他们最初的构造受限于处理由阈值门组成的
公式。\par
在随后的工作中,Goyal, Pandey, Sahai, and Waters\cite{GPSW:ABE}
进一步阐明了ABE的概念。尤其是,他们提出了两种互补形式的ABE。
其一是密钥策略--基于属性的加密\footnote{Key-Policy ABE,简称KP-ABE},
属性用来描述密文,对属性的公式用来描述用户的私钥。其二是与KP-ABE
互补的密文策略--基于属性的加密\footnote{Ciphertex-Policy ABE,简称
CP-ABE},属性用来描述用户的凭据即私钥,公式则被加密放附加到密文。
除此之外。Goyal等人\cite{GPSW:ABE}提供了KP-ABE的一种构造。由于它
允许被加密数据的密钥能够被任意单调访问公式描述,所以该构造的代
价较高。这个系统被证明在双线型Diffie-Hellma假设下选择性安全。然
而他们把CP-ABE描述性的构造留做了一个开放性问题。\par
第一次明确解决CP-ABE问题的是Bethencourt,Sahia和Waters\cite{BSW:CPABE}。
他们描述了一个高效的可以描述的系统,这个系统允许加密者以任意单调
访问公式来表达访问谓词$f$,并取得了Gyal等人\cite{GPSW:ABE}系统的
类似表现和效率。
\section{本文的主要工作}
本文主要描述BSW的工作\cite{BSW:CPABE}。在BSW的系统中用户的私钥与
任意数目的字符串形式的属性相关联。另一方面,当一方要要加密数据时,
他们明确相关的属性上的访问结构。仅当用户的属性可以通过密文的访问
结构,他才能解密密文。从数学角度讲,访问结构被描述为单调“访问树”,
访问结构的节点由阈值门组成,叶子节点描述了属性。我们用\textbf{AND}
来构造$n\textrm{-}of\textrm{-}n$阈值门,用\textbf{OR}表示$1\textrm{-}of\textrm{-}n$阈值门。
\section{本文的主要技术}
从较高的水平上讲,本文的主要工作与最近Sahai和Waters\cite{SW:FuzzyIBE},
Goyal等人\cite{GPSW:ABE}的工作相似,但是本文却大量采用了新的技术。
KP-ABE,密文与描述性的属性关联,用户的私钥与策略关联(与我们的情况
恰好相反)。\textit{我们强调KP-ABE中,加密者对谁可以访问他所加密的
数据不具有控制权,除非他选择对数据的描述性数据。}然而,他必须保证
提供给用户适当的私钥以授权或者禁止用户解密密文。换言之,在
\cite{SW:FuzzyIBE,GPSW:ABE}中,“情报”掌握在私钥询问者手中,而不是
加密者手中。在我们的构造里,加密者决定了其他人能否访问他所加密的
数据。因此,在\cite{SW:FuzzyIBE,GPSW:ABE}中的技术不适合我们,我们
必须采用新的技术。\par
从技术的水平上讲,我们必须达到的主要目标是\textit{可抵抗合谋攻击}:
如果多个用户合谋,他们只能解密至少他们中的一个用户可以独立解密的
密文。特别提到绪论一开始提到的例子,假设工作在San Francisco反恐局
的FBI特工与他工作在New York公共反腐局的朋友合谋。我们不希望合谋者
能够解密由他们共同属性加密的备忘录。在我们的构造中,这种安全是
\textit{先决条件}。\par
在\cite{SW:FuzzyIBE,GPSW:ABE}的工作中,通过利用秘密分享方案
\footnote{Secret-Sharing Scheme} \cite{Shamir:SSS,Blakley:SSS}和独
立地选择私密份额嵌入到每一个人的私钥中来保证可地址合谋攻击。秘密分
享方案的每次调用中随机数是独立的,因此可抵制合谋攻击。在我们的情况
中,用户的私钥与属性集而不是属性集上的访问结构关联,因此秘密分享方
案不适用。\par
相反,我们设计了一种采用两级随机掩蔽技术的新的私钥随机技术。这项技
术利用了高效可计算的双线性映射群。\par
最后,我们提供了我们构造的一个实现来展示实践中我们的系统表现是良好
的。
\section{本文的组织结构}
本文的余下部分组织结构如下。在第二章我们将讨论相关的工作,第三章我
们将给出我们的定义以及高效可计算双线性映射群的相关概念,第四章我们
将讨论BSW的构造,第五章我们会提供一个CP-ABE的Java实现,第六章将是本
文的总结。

% 相关工作
%%%%%%%%%%%%%%%%%%%%%%%%%%%%%%%%%%%%%%%%%%%%%%%%%%%%%%
% Filename:      related_work.tex
% Author:        Junwei Wang(wakemecn@gmail.com)
% Last Modified: 2012-04-17 10:13
% Description:
%%%%%%%%%%%%%%%%%%%%%%%%%%%%%%%%%%%%%%%%%%%%%%%%%%%%%%
%\chapter{相关工作}
%\section{IBE及其扩展}
%Sahai和Waters提出了ABE的概念的同时,提出了一个叫做模糊的基于身份的加密
%\footnote{Fuzzy-IBE,简称FIBE}的特别系统。FIBE机制建立在一些IBE思想的基础
%\cite{BF:IBE,Shamir:IBE,Cocks:IBE,CHK:fspk,BB:IBE}之上。在FIBE中,身份被
%看作是属性的集合。如果私钥拥有身份$\omega$,密文基于身份$\omega'$,当且仅
%当$\omega$和$\omega'$“设置重叠”的距离度量足够接近私钥才能解密密文。换言之,
%如果消息用属性集合$\omega'$加密,私钥由属性集合$\omega$构成,当且仅当
%$|\omega\cap\omega'|\geq d$\footnote{d在加密是已经确定}时,私钥才能解密密
%文。因此,FIBE实现了容错并适用于生物身份识别。然而,FIBE却不适用我们本文
%的主要目的即数据的访问控制。由于FIBE的主要目标是实现容错,因此它唯一支持
%的访问结构是阈值在加密时已经确定的阈值门。\par
%

% 背景 
%%%%%%%%%%%%%%%%%%%%%%%%%%%%%%%%%%%%%%%%%%%%%%%%%%%%%%
% Filename:      background.tex
% Author:        Junwei Wang(wakemecn@gmail.com)
% Last Modified: 2012-04-21 22:01
% Description:
%%%%%%%%%%%%%%%%%%%%%%%%%%%%%%%%%%%%%%%%%%%%%%%%%%%%%%
\chapter{背景}
我们首先给出CP-ABE安全性的正式定义,然后我们给出双线性映射的背景信息。如同Goyal等人
\cite{GPSW:ABE}所做,我们定义访问结构并在我们的安全性定义中使用;不同的是,在我们的
定义中属性将用来描述用户,访问结构用来标记被加密的数据。
%%%%%%%%%%%%%%%%%%%%%%%%%%%%%%%%%%%%%%%%%%%%%%%%%%%%%%%
%%%%%%%%%%%%%%%%%%%%%%%%%%%%%%%%%%%%%%%%%%%%%%%%%%%%%%%
\section{定义}
\begin{Definition}[访问结构\cite{Beimel:SSSS}]
令$\{P_1,P_2,...,P_n\}$是所有的参与方集合。集合$\mathbb{A}\subseteq 2^{\{P_1,P_2,...,P_n\}}$
是单调的是指:对于任意$B,C$,如果$B\in\mathbb{A}$且$B\subseteq C$,则$C\in\mathbb{A}$。
访问结构(特别的,单调访问结构)是$\{P_1,P_2,...,P_n\}$的非空子集,例如,
$\mathbb{A}\subseteq 2^{\{P_1,P_2,...,P_n\}}\setminus\{\emptyset\}$,
的集合$\mathbb{A}$(特别的,单调集合)。集合$\mathbb{A}$中的集合被称为授权集,不在
集合$\mathbb{A}$中的集合被称为未授权集。
\end{Definition}
\par
在我们的论述中,参与方被属性所代替,因此,访问结构$\mathbb{A}$包含了授权属性集。本文中
我们只注意单调访问结构,但是,以我们的技术实现通用的访问结构是可能的。从现在开始,除非
特别强调,访问结构是指单调访问结构。\par
%%%%%%%%%%%%%%%%%%%%%%%%%%%%%%%%
\subsection{CP-ABE方案}
一个CP-ABE机制包含四个基本的算法:Setup,Encrypt,KeyGen和Decrypt。\\
\textbf{Setup}\quad Setup算法的输入只有隐含的安全参数,输出公共参数PK和主密钥MK。\\
\textbf{Encrypt(Pk,$M$,$\mathbb{A}$)}\quad 加密算法的输入是公共参数PK,明文$M$和属性全集
上的访问结构$\mathbb{A}$。算法加密明文$M$,产生密文CT,只有用户持有满足访问结构的属性
集合才能解密消息。我们假定密文隐含访问结构$\mathbb{A}$。\\
\textbf{Key Generation(MK,$S$)}\quad
密钥生成算法的输入是主密钥MK和描述私钥的属性集合$S$,输出是私钥SK。\\
\textbf{Decrypt(PK,CT,SK)}\quad
解密算法的输入是公共参数PK,包含访问策略$\mathbb{A}$的密文CT和属性集合$S$生成的私钥SK。
如果集合$S$满足访问结构$\mathbb{A}$那么算法会解密密文CT返回消息$M$。\par
%%%%%%%%%%%%%%%%%%%%%%%%%%%%%%%%
现在我们将描述CP-ABE方案的安全模型。如同IBE方案\cite{Shamir:IBC,BF:IBE,Cocks:IBE}的安全
模型,CP-ABE方案的安全模型允许敌手询问任何比能解密密文的私钥,即在我们的安全定义中敌手
挑战访问结构$\mathbb{A}^{*}$的解密,并且可以询问不满足$\mathbb{S}^{*}$的私钥$S$。
%%%%%%%%%%%%%%%%%%%%%%%%%%%%%%%%
\subsection{CP-ABE安全模型}
现在我们给出正式的安全游戏。\\
\textbf{Setup}\quad 挑战者运行Setup算法并将得到的公共参数PK交给敌手。\\
\textbf{阶段1}\quad 敌手反复生成对应于属性集$S_1,...,S_{q_1}$的私钥。\\
\textbf{Challenge}\quad 敌手提供两个等长的信息$M_0$和$M_1$。除此之外,敌手提供一个阶段1
中所有属性集合$S_1,...,S_{q1}$都不满足的挑战访问结构$\mathbb{A^*}$。挑战者随机选取
$b\in{0,1}$,在访问结构$\mathbb{A^*}$下加密$M_b$,并将密文$CT^*$敌手。\\
\textbf{阶段2}\quad 在属性集合$S_{q_1},...,S_q$不能满足对应的挑战访问结构的限制下重复
阶段1。\\
\textbf{Guess}\quad 敌手输出对$b$的猜测$b'$。\par
%%%%%%%%%%%%%%%%%%%%%%%%%%%%%%%%
\vspace{3mm}
在上述游戏中敌手$\mathcal{A}$的优势被定义为$\mathrm{Pr}[b'=b]-\frac{1}{2}$。我们注意到这个模型可以通过在阶段1和阶段2中允许解密敌手询问的明文来扩展为可以处理选择明文攻击的情形。
\begin{Definition}
在上述游戏中,如果多项式时间内敌手最多有可以忽略的优势,那么CP-ABE方案是安全的。
\end{Definition}
%%%%%%%%%%%%%%%%%%%%%%%%%%%%%%%%%%%%%%%%%%%%%%%%%%%%%%%
%%%%%%%%%%%%%%%%%%%%%%%%%%%%%%%%%%%%%%%%%%%%%%%%%%%%%%%
\section{双线性映射}
我们提出一些与群相关的高效可计算双线型映射的事实。\par
设$\mathbb{G}_0$和$\mathbb{G}_1$是阶为素数$p$的两个乘法循环群,$g$是$\mathbb{G}_0$
的生成元,$e:\mathbb{G}_0\times \mathbb{G}_0\to \mathbb{G}_1$。双线性映射$e$具有以
下性质:
\begin{enumerate}
\setlength{\itemsep}{0pt}
\item 双线性:对于任意$u,v\in \mathbb{G}_0$和任意$a,b\in \mathbb{Z}_p$,我们有
$e(u^a,v^b) =e(u,v)^{ab}$
\item 非退化性:$e(g,g)\neq 1$
\end{enumerate}
\par
如果$\mathbb{G}_0$中的群操作和双线性映射
$e:\mathbb{G}_0\times \mathbb{G}_0\to\mathbb{G}_1$都是高效可计算的,我们称
$\mathbb{G}_0$为双线性群。请注意由于$e(g^a,g^b)=e(g,g)^{ab}=e(g^b,g^a)$,
双线性映射$e$具有对称性。

% 构造
%%%%%%%%%%%%%%%%%%%%%%%%%%%%%%%%%%%%%%%%%%%%%%%%%%%%%%
% Filename:      construction.tex
% Author:        Junwei Wang(wakemecn@gmail.com)
% Last Modified: 2012-05-03 10:28
% Description:
%%%%%%%%%%%%%%%%%%%%%%%%%%%%%%%%%%%%%%%%%%%%%%%%%%%%%%
\chapter{CP-ABE构造}
本章中我们提供了BSW方案的构造。我们首先介绍描述密文的访问树和描述私钥的属性的模型,
其次我们给出我们方案的描述,最后我们给出安全性,高效性的讨论。
%%%%%%%%%%%%%%%%%%%%%%%%%%%%%%%%%%%%%%%%%%%%%%%%%%%%%%
%%%%%%%%%%%%%%%%%%%%%%%%%%%%%%%%%%%%%%%%%%%%%%%%%%%%%%
\section{模型}
在我们的构造中私钥被描述性的属性集合$S$所确定。一方想解密数据必须拥有满足访问树结构的
私钥。\par
访问树中的每个内部节点是一个阈值门,叶子节点与属性关联。(请注意,这样的描述是非常
具有表达力的。例如,我们分别用2 of 2和1 of 2的阈值门来表达树中“AND”和“OR”门。)当且
仅当私钥的属性分配到树的节点并使树能满足,持有私钥的用户将能够解密密文。尽管在我们的
方案里,属性用来标记私钥,我们还是用与\cite{GPSW:ABE}相同的概念来描述访问树。\par
\vspace{5mm}
\noindent\textbf{访问树$\mathcal{T}$}\quad 用$\mathcal{T}$代表访问结构,用通过他的孩
子和阈值描述阈值门来代表树的每个非叶子节点。设节点$x$的孩子数目为$num_x$,阈值为$k_x$,
那么$0<k_x\leq num_x$。当$k_x=1$,阈值门是或门;当$k_x=num_x$,阈值门是与门。树的每个
叶子节点$x$用一个属性和阈值$k_x=1$描述。\par
为了方便的使用访问树,我们定义了一些函数。函数$\textbf{parent}(x)$表示节点$x$的
父节点,仅当$x$是叶子节点时,函数$\textbf{att}(x)$被定义为$x$所描述的属性。访问树
$\mathcal{T}$同时定义了每个节点的子节点的被标记为从1到$num$的顺序索引值,函数
$\textbf{index}(x)$定义为与节点$x$关联的这样一个索引值,且对一个给定的密钥,索引值
以任意的方式唯一的赋值给访问结构中的节点。\par
\vspace{5mm}
\noindent\textbf{满足访问树}\quad 令$\mathcal{T}$是根为$r$的访问树,$\mathcal{T}_x$
表示以$x$为根的$\mathcal{T}$的子树,因此$\mathcal{T}$等同于$\mathcal{T}_r$。如果属性
集合$\gamma$满足访问树$\mathcal{T}_x$,我们表示为$\mathcal{T}_x(\gamma)=1$。我们用如
下递归的方式计算$\mathcal{T}_x(\gamma)$:如果$x$是非叶子节点,对于$x$的所有孩子,
计算$\mathcal{T}_{x'}(r)$,当且仅当至少$k_x$个孩子返回1,$\mathcal{T}_x(\gamma)$返回1;
如果$x$是叶子节点,那么当且仅当$\textbf{attr}(x)\in \gamma$时,$\mathcal{T}_x(\gamma)$
返回1。
%%%%%%%%%%%%%%%%%%%%%%%%%%%%%%%%%%%%%%%%%%%%%%%%%%%%%%
%%%%%%%%%%%%%%%%%%%%%%%%%%%%%%%%%%%%%%%%%%%%%%%%%%%%%%
\section{构造}
设$\mathbb{G}_0$是阶为素数$p$的双线性映射,$g$是$\mathbb{G}_0$的生成元,
$e:\mathbb{G}_0\times \mathbb{G}_0\to\mathbb{G}_1$是双线性映射。安全参数$\kappa$决定
群的规模。同时,我们定义了拉格朗日系数$\Delta_{i,S}$,$i\in\mathbb{Z}_p$,$S$是集合
$\mathbb{Z}_p$中的元素:$\Delta_{i,S}(x)=\prod_{j\in S,j\neq i}\frac{x-j}{i-j}$。
哈希函数$H:{0,1}^*\to \mathbb{G}_0$是random oracle,这个函数可以把任意用二进制串
描述的的属性映射到随机的群元素。我们的构造如下。\par
%%%%%%%%%%%%%%%%%%%%%%%%%%%%
\vspace{5mm}
\noindent\textbf{Setup}\quad 
Setup算法选择生成元为$g$,阶为素数$p$的双线性映射
$\mathbb{G}_0$,指数$\alpha,\beta\in\mathbb{Z}_p$。生成公钥
$$\textrm{PK} = \mathbb{G}_0,g,h=g^{\beta},f=g^{1/\beta},e(g,g)^{\alpha}$$
和主密钥$\textrm{MK} = \beta,g^\alpha$。(注意$f$仅用于委托。)\par
%%%%%%%%%%%%%%%%%%%%%%%%%%%%
\vspace{5mm}
\noindent\textbf{Encryption(PK,$M$,$\mathcal{T}$)}\quad
加密算法在树访问结构$\mathcal{T}$下加密消息$M$。算法首先为$\mathcal{T}$每个节点$x$
(包括叶子节点)选择一个多项式$q_x$。从树的根节点$R$开始,自上而下选择多项式。节点
$x$的多项式$q_x$的度$d_x$比该节点的阈值$k_x$少一,即$d_x=k_x-1$。\par
算法从根节点$R$开始选择随机数$s\in\mathbb{Z}_p$,并设置$q_R(0)=s$。然后,算法随机
选择多项式$q_R$上的$d_R$个点来完全定义$q_R$。对于其他的顶点$x$,令
$q_x(0)=q_{parent(x)}(index(x))$,随机选择其它$d_x$个顶点来完全定义$q_x$。\par
设$\mathcal{T}$中所有叶子节点的集合为$Y$,那么在给定的树形访问结构$\mathcal{T}$下计算
密文
\begin{equation*}
\begin{split}
\textrm{CT}=&(\mathcal{T},\widetilde{C}=Me(g,g)^{\alpha s},C=h^s,\\
&\forall y\in Y:\quad C_y=g^{q_y(0)},C_y'=H(\textbf{att}(y))^{q_y(0)}).
\end{split}
\end{equation*}
%%%%%%%%%%%%%%%%%%%%%%%%%%%%
\vspace{5mm}
\noindent\textbf{KeyGen(MK,$S$)}\quad
密钥生成算法的输入是属性集合$S$,输出为被$S$所标记的密钥。算法首先选择随机数
$r\in \mathcal{Z}_p$,然后对每一个$j\in S$选择随机数$r_j\in \mathcal{Z}_p$。最后计算出
私钥
\begin{equation*}
\begin{split}
\textrm{SK}=&(D=g^{(\alpha+\gamma)/\beta},\\
&\forall j\in S:\quad D_j=g^r\cdot H(k)^{rj},D_j'=g^{rj}).
\end{split}
\end{equation*}
%%%%%%%%%%%%%%%%%%%%%%%%%%%%
\vspace{5mm}
\noindent\textbf{Decrypt(CT,SK)}\quad
我们的解密算法是一个递归算法。为了便于讨论,我们提出了解密算法的最简单形式,并在下一小
节中提出了潜在的性能改进。\par
我们首先定义递归算法Decrypt(CT,SK,$x$),它用密文
$\textrm{CT}=(\mathcal{T},\widetilde{C},C,\forall y\in Y:\quad C_y,C_y')$,与属性集合
$S$关联的私钥SK,$\mathcal{T}$中的节点$x$作为输入。\par
当节点$x$是叶子节点,令$i=\textbf{att}(x)$,如果$i\in S$,那么
\begin{equation*}
\begin{split}
DecryptNode(CT,SK,x) &= \frac{e(D_i,C_x)}{e(D_i',C_x')}\\
&=\frac{e(g^r\cdot H(i)^{r_i},g^{q_x(0)})}{e(g^{r_i},H(i)^{q_x(0)})}\\
&=e(g,g)^{rq_x(0)}.
\end{split}
\end{equation*}
如果$i\notin S$,那么DecryptNode(CT,SK,$x$)$=\perp$。\par
现在我们考虑$x$是非叶子节点时的递归情况。算法DecryptNode(CT,SK,$x$)工作方式如下:
对于$x$的所有字节点$z$,计算$F_z=\textrm{DecryptNode(CT,SK,}z\textrm{)}$。
令$S_x$为$k_x$大小的满足$F_z\neq \perp$的子节点$z$的集合。如果不存在这样的集合,
那么这个节点不满足,且函数返回$\perp$。\par
否则,我们计算
\begin{equation*}
\begin{split}
F_x &=\prod_{z\in S_x}F_x^{\Delta_{i,S_x'(0)}}
\quad \text{,其中,$i=index(z),S_x'=\{index(z),z\in S_x\}$}\\
&=\prod_{z\in S_x}(e(g,g)^{r\cdot q_z(0)})^{\Delta_{i,S_x'(0)}}\\
&=\prod_{z\in S_x}(e(g,g)^{r\cdot q_{parent(z)}(index(z))})^{\Delta_{i,S_x'(0)}}\\
&=\prod_{z\in S_x}e(g,g)^{r\cdot q_z(0)\cdot \Delta_{i,S_x'(0)}}\\
&=e(g,g)^{r\cdot q_x(0)} \qquad \text{(使用多项式插值)}
\end{split}
\end{equation*}
并返回结果。\par
在定义了DecryptNode函数之后,我们定义解密算法。算法首先调用Decrypt(CT,SK,$R$),
$R$是树$\mathcal{T}$的根节点。如果树满足$S$,我们令
$$A=\textrm{DecryptNode(CT,SK,}R\textrm{)}=e(g,g)^{rq_R(0)}=e(g,g)^{rs}$$.
现在算法通过下面计算解密
$$\widetilde{C}/\left(\frac{e(C,D)}{A}\right)=
\widetilde{C}/\left(\frac{e(h^s,g^{(\alpha+\gamma)/\beta})}{e(g,g)^{rs}}\right)=M$$
%%%%%%%%%%%%%%%%%%%%%%%%%%%%%%%%%%%%%%%%%%%%%%%%%%%%%%
%%%%%%%%%%%%%%%%%%%%%%%%%%%%%%%%%%%%%%%%%%%%%%%%%%%%%%
\section{直觉安全和效率}
现在我们将提出我们方案的直觉安全性和我们方案的效率的讨论。
\subsection{直觉安全}
如同之前的ABE方案,在设计我们的方案时面临的最大挑战是如何抵制合谋攻击。我们采用了
Sahai和Waters方案\cite{SW:FuzzyIBE}中的通过随机化用户私钥的方案来组织合谋;所不同的是,
我们把秘密分享的思想嵌入到密文而不是私钥中。攻击者必须恢复$e(g,g)^{\alpha s}$才能够
解密。为了做到这一点,攻击者必须从用户的私钥中配对$C$,$D$两部分。这回得到想要的值
$e(g,g)^{\alpha s}$,但是被某些值$e(g,g)^{rs}$所蒙蔽。当且进当某用户拥有密钥的部分满足
嵌入到密文中的秘密分享,这个值才不会被蒙蔽。合谋攻击不会在上述过程中起到任何作用,因为
用户私钥的随机性随机化了盲值。\par
尽管我们的方案在选择明文攻击情况下是安全的,我们方案可以高效的扩展为选择密文攻击。
这个过程需要想Fujisaki-Okamoto变换\cite{FO}一样采用random oracle工具。
\subsection{效率}
密钥生成算法和加密算法的效率都非常容易计算。加密算法对密文访问树每个叶子节点需要两次
幂运算,密文的规模中包括每个树叶子节点的两个群元素。密钥生成算法对用户的每个属性需要
两次指数运算,私钥包含对每个属性的两个群元素。最简单的形式,解密算法需要对与私钥属性
匹配的访问树叶子节点两次配对,对沿着上述叶子到根节点的路径节点的
(至多\footnote{如果路径中存在不满足的内部节点,幂运算会减少})一次指数元算。然而,
可能存在着不止一种满足策略的方式,一个更加智能的算法可能会沿着这条线路优化,我们
会在第四章中介绍多种提升性能的方法。

% 实现
%%%%%%%%%%%%%%%%%%%%%%%%%%%%%%%%%%%%%%%%%%%%%%%%%%%%%%
% Filename:      implementation.tex
% Author:        Junwei Wang(wakemecn@gmail.com)
% Last Modified: 2012-05-05 10:28
% Description:
%%%%%%%%%%%%%%%%%%%%%%%%%%%%%%%%%%%%%%%%%%%%%%%%%%%%%%
\chapter{实现}
在本章中我们将要讨论实现第三章构造的一些实际问题,包括一些优化,我们所开发的工具包
的描述以及性能的测量。
\section{解密算法的效率改进}
尽管没有必要通过新的技术减少setup、密钥生成和加密算法的群运算,但是这却可以大幅提升
解密算法的性能。我们将在次介绍这些改进,并在4.3节中给出他们的作用。\par
%%%%%%%%%%%%%%%%%%%%%%%%%%%%%%%%%%%%%%%%%%%%%%%%%%%%%%%%%%%%%%%%%%%%%%%%
\vspace{5mm}
\textbf{优化解密策略}\quad 第三章出的递归算法中,被对私钥属性匹配的叶子节点进行两次
配对,对沿着节点到根的每个内部节点(不包括根节点)至多一次指数运算。递归部分的最后一
步会有一次额外的配对。当然,每个阈值为$k$的内部节点,只保留他的$k$个孩子。考虑到在
提前叶节点满足和选择可以满足整个访问树的自己的时间,我们可能避免估计DecryptNode,因为
结果最终不被使用。\par
更精确的讲,设$M$是访问树$\mathcal{T}$的节点的子集。我们定义函数$\mathbf{restrict}(
\mathcal{T},M)$是从$\mathcal{T}$删除以下节点(同时保持阈值不变)形成的访问树。
首先,我们删除所有不在$M$中的节点,下一步我们删除沿着孩子数小于阈值$k_x$的内部节点
$x$一直到$\mathcal{T}$的根节点的所有节点。重复上述步骤直至没有节点可以删除,这时侯
得到的就是$\mathbf{restrict}(\mathcal{T},M)$。因此给定访问树$\mathcal{T}$和满足访问树
的属性集$\gamma$,自然的问题是选择集合$M$使得$\gamma$满足$\mathbf{restrict}(
\mathcal{T},M)$并且$M$中的叶子数目是最少的(考虑到配对操作的代价是最昂贵的)。这可以通过对树的一次遍历的递归算法很容易的实现。算法DecryptNode在
$\mathbf{restrict}(\mathcal{T},M)$上会取得与原来相同的结构。\par
%%%%%%%%%%%%%%%%%%%%%%%%%%%%%%%%%%%%%%%%%%%%%%%%%%%%%%%%%%%%%%%%%%%%%%%%
\vspace{5mm}
\textbf{直接计算}\quad
进一步性能改进,可以放弃DecryptNode函数并采取更直接的计算。直观的讲,我们想象展平
树上对DecrypNode的递归调用,然后合并指数运算到每一个(用过的)叶子节点上。更精确的讲,
令$\mathcal{T}$是根为$r$的访问树,$\gamma$是属性集合,$M \subseteq \mathcal{T}$使得
$\gamma$满足$\mathbf{restrict}(\mathcal{T},M)$。同时假设$M$是最小的,即没有内部节点的
孩子数比阈值大。设$L \subseteq M$是$M$中的叶子节点。那么对于每一个$l \in L$,
$l$到$r$的路径表示为
$$\rho(l)=(l,\mathbf{parent}(l),\mathbf{parent}(\mathbf{parent}(l)),...r) .$$
同时,节点$x$的兄弟节点(包括$x$)表示为
$$\mathbf{sibs}(x)=\{y\mid \mathbf{parent}(x)=\mathbf{parent}(y)\}\quad.$$
在上述概念的基础上,我们可以继续直接计算Decrypt(CT,SK,$r$)的结果。首先,对每个
$l\in L$计算
\begin{align*}
z_l = \prod_{x\in \rho(l)\atop x\neq r}\quad \text{其中$i=\mathbf{index}(x),
S=\{\mathbf{y}|y\in\mathbf{sibs}(x)\}$ ,}
\end{align*}
然后计算
$$\mathrm{DecryptNode(CT,SK,}r\mathrm{)}
=\prod_{l\in L\atop i=\mathbf{att}(l)}
\left(\frac{e(D_i,C_l)}{e(D_i',C_l')}\right)^{z_l}\quad .$$
用这种方法,整个解密算法中指数运算的次数从$M-1$(例如,除去根节点以外的每个节点一次)
减少为$|L|$,配对的次数为$2|L|$。\par
%%%%%%%%%%%%%%%%%%%%%%%%%%%%%%%%%%%%%%%%%%%%%%%%%%%%%%%%%%%%%%%%%%%%%%%%
\vspace{5mm}
\textbf{合并配对}\quad
合并具有相同属性的叶子可能进一步较少配对的数目。如果对于$L$中的某些$l_1,l_2$,有
$\mathbf{att}(l_1)=\mathbf{att}(l_2)=i$
,那么
\begin{equation*}
\begin{split}
&\left(\frac{e(D_i,C_{l_1})}{e(D_i',C_{l_1}')}\right)^{z_{l_1}}
\cdot\left(\frac{e(D_i,C_{l_2})}{e(D_i',C_{l_2}')}\right)^{z_{l_2}}\\
=&\frac{e(D_i,C_{l_1}^{z_{l_1}})}{e(D_i',C_{l_1}^{z_{l_1}})}
\cdot\frac{e(D_i,{C'}_{l_2}^{z_{l_2}})}{e(D_i',{C'}_{l_2}^{z_{l_2}})}\\
=&\frac{e(D_i,C_{l_1}^{z_{l_1}}\cdot C_{l_2}^{z_{l_2}})}
{e(D_i,{C'}_{l_1}^{z_{l_1}}\cdot {C'}_{l_2}^{z_{l_2}})}\quad .
\end{split}
\end{equation*}
利用这个事实,我们合并$L$中所有互相不同的属性的配对,配对总数减少至$2m$,$m$是出现在
$L$中不同的属性的个数。但是请注意,幂运算的次数增加了,且某些幂运算现在执行在
$\mathbb{G}_0$上而不是$\mathbb{G}_1$上。具体来讲,如果$m'$是至少与其他叶子节点分享属性
的叶子节点的数目,那么我们在$\mathbb{G}_0$和$\mathbb{G}_1$上分别执行$2m'$次,$|L|-m'$
次指数运算,而不是分别为$0$次,$|L|$次。如果椭圆曲线群$\mathbb{G}_0$上的幂运算比与其
阶相同的有限域$\mathbb{G}_1$慢,那么这个技术将潜在的增加解密的时间。我们将在4.3节讨论
这个权衡。
\section{\texttt{cpabe}使用范例}
我们给出了第三章构造的实现,称之为\texttt{cpabe}包\cite{cpabe},并在GPL协议\cite{gpl}
下发布,实现使用了jPBC库\cite{jPBC}\footnote{jPBC遵循LGPL协议发布}。\texttt{cpabe}包
的可以点用CPABE类的如下成员函数的方式直接被其它大型的系统调用。
\lstinputlisting[caption=Setup]{Setup.java}
\lstinputlisting[caption=Keygen]{Keygen.java}
\lstinputlisting[caption=Enc]{Enc.java}
\lstinputlisting[caption=Dec]{Dec.java}
附录A包含一个具体的调用实例。







% 附录
%%%%%%%%%%%%%%%%%%%%%%%%%%%%%%%%%%%%%%%%%%%%%%%%%%%%%%
% Filename:      appendix.tex
% Author:        Junwei Wang(wakemecn@gmail.com)
% Last Modified: 2012-05-05 10:47
% Description:
%%%%%%%%%%%%%%%%%%%%%%%%%%%%%%%%%%%%%%%%%%%%%%%%%%%%%%
\titleformat{\chapter}[display]{\centering\normalfont\Large\bfseries}
{附录~\Alph{chapter}}{11pt}{\Large}
\renewcommand{\thechapter}{附录\Alph{chapter}.}
\begin{appendices}
\chapter{CPABE使用演示}
\lstinputlisting[caption=DemoForCpabe.java]{DemoForCpabe.java}
\end{appendices}


\addcontentsline{toc}{chapter}{\qquad 参考文献}
\bibliographystyle{plain}
\bibliography{graduation_design}
\end{document}
